%\part{Aspectos Gerais}

\chapter[Proposta Inicial]{Proposta Inicial}
	
	Durante esta seção serão apresentadas questões referentes à proposta de tema do Trabalho de Conclusão de Curso de Engenharia de Software.

\section{Contextualização}

	Quando se pensa em robótica, automaticamente nos vêm a cabeça o conceito de automação. Robôs possuem a necessidade de serem autônomos, principalmente os robôs móveis, que precisam se locomover no ambiente sem o apoio de um controlador humano.

	As três leis fundamentais da robótica levam em consideração um robô totalmente autônomo, que seria capaz de realizar qualquer atividade sem o apoio de um humano. Uma das questões mais importantes quando se trata da autonomia dos robôs é referente a mobilidade dos mesmos. Segundo \cite{localizacaoEMapeamentoPaulo} a autonomia de um robô é fortemente condicionada pela sua capacidade de perceber o ambiente de navegação, interagindo com o meio e realizando tarefas com o mínimo de precisão. Este mínimo, segundo \cite{localizacaoEMapeamentoPaulo}, seria a navegação sem colisão com obstáculos. 

	Para que robôs sejam capazes de navegar em um ambiente desconhecido sem que haja colisão em objetos e obstáculos, os mesmos necessitam de informações sobre este ambiente. Estas informações são adquiridas utilizando sensores. Como foi apresentado por \cite{interacaoRoboAmbiente}, no livro de Robótica Industrial, os sensores possuem o dever de fornecer informações ao sistema de controle do robô sobre distâncias de objetos, posição do robô, contato do robô com objetos, força exercida sobre objetos, cor e textura dos objetos, entre outras.

	Além de observar e obter informações sobre o ambiente, o robô precisa se auto-localizar para processar as informações obtidas e traçar rotas sem colisões até o ponto de destino. Para isso, foram desenvolvidas muitas formas de auto-localização, algumas delas são citadas por \cite{roboBulldozerIV}, como:

	\begin{itemize}
		\item \textbf{Utilização de Mapas}: O robô conhece o mapa onde realizará a navegação à priori, conhecendo os obstáculos e os caminhos possíveis. Possuindo essas informações, o robô irá traçar as rotas mais eficientes para chegar em seu objetivo.

		\item \textbf{Localização Relativa em Grupos}: Está técnica utiliza a navegação simultânea de muitos robôs, cada robô sabe a posição relativa dos outros robôs, podendo calcular sua posição relativa.

		\item \textbf{Utilização de Pontos de Referência}: Conhecendo pontos de referência que estão distribuídos pelo mapa de navegação, o robô consegue calcular sua posição através da técnica de triangulação.

		\item \textbf{Localização Absoluta com GPS}: A partir desta técnica é fácil obter a posição absoluta do robô em relação a terra. O grande problema desta técnica é a margem de erro presente no sistema de GPS, inviável para navegações internas.

		\item \textbf{Utilização de Bússolas}: É uma técnica interessante para conhecimento da orientação do robô, o que facilita muito na navegação do mesmo. Porém as bússolas são muito frágeis a interferências externas, como por exemplo a proximidade de materiais ferro-magnéticos ou as fugas magnéticas dos motores presentes no próprio robô.

		\item \textbf{Odometria}: Consiste na medição da distância relativa percorrida pelo robô, utilizando sensores presentes nas rodas do mesmo. Necessita do conhecimento do ponto de origem.
		 
	\end{itemize}

	Levando em consideração a primeira técnica apresentada, a utilização de mapas, \cite{localizacaoEMapeamentoPaulo} mostra a possibilidade do próprio robô construir o mapa do ambiente e utilizá-lo para navegação, simultaneamente. Este problema é conhecido como problema SLAM (da sigla em inglês), que consiste em um robô autônomo iniciar a navegação em uma localização desconhecida e  ambiente desconhecido, construindo um mapa deste ambiente de forma incremental e utilizando-o para calcular sua localização atual.

	A utilização do SLAM é bastante útil quando não exite o conhecimento prévio do ambiente onde se deseja navegar. Segundo \cite{localizacaoEMapeamentoPaulo}, a resolução do problema SLAM pode ocorrer a partir de diferentes técnicas.