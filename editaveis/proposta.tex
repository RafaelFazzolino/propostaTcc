%\part{Aspectos Gerais}

\chapter[Proposta Inicial]{Proposta Inicial}

\section{Contextualização}
	
	O nascimento da robótica se deu no contexto industrial, onde ferramentas autônomas foram desenvolvidas para executar atividades de forma repetitiva e incansável, maximizando a qualidade dos produtos produzidos e minimizando o custo e tempo para produção dos mesmos. A palavra robô é derivada da palavra \textit{robota}, de origem eslava, que significa \textit{trabalho forçado}. Enquanto a robótica se mantinha no contexto industrial, não havia grande necessidade de implementar robôs com autonomia em relação a sua mobilidade. A utilização de esteiras, por exemplo, era suficiente para solucionar o problema da locomoção dos robôs em chão de fábrica. Porém, com a ampliação das áreas de atuação destas ferramentas autônomas, surge a necessidade da autonomia relacionada a mobilidade das mesmas. (REFERENCIAR ESSE PARÁGRAFO TODO)

	Segundo \cite{localizacaoEMapeamentoPaulo}, a autonomia de um robô é fortemente condicionada pela sua capacidade de perceber o ambiente de navegação, interagindo com o meio e realizando tarefas com o mínimo de precisão. Este mínimo, segundo \cite{localizacaoEMapeamentoPaulo}, seria a navegação sem colisão em obstáculos. 

	Para que robôs sejam capazes de navegar em um ambiente desconhecido sem que haja colisão em objetos e obstáculos, os mesmos necessitam de informações sobre este ambiente. Estas informações são adquiridas utilizando sensores. Como foi apresentado por \cite{interacaoRoboAmbiente}, no livro de Robótica Industrial, os sensores possuem o dever de fornecer informações ao sistema de controle do robô sobre distâncias de objetos, posição do robô, contato do robô com objetos, força exercida sobre objetos, cor e textura dos objetos, entre outras.

	Além de obter informações sobre o ambiente, o robô precisa se auto-localizar para processar as informações obtidas e traçar rotas sem colisões até o ponto de destino. Para isso, foram desenvolvidas muitas formas de auto-localização, algumas delas são citadas por \cite{roboBulldozerIV}, como:

	\begin{itemize}
		\item \textbf{Utilização de Mapas}: O robô conhece o mapa onde realizará a navegação à priori, conhecendo os obstáculos e os caminhos possíveis. Possuindo essas informações, o robô irá traçar as rotas mais eficientes para chegar em seu objetivo.

		\item \textbf{Localização Relativa em Grupos}: Esta técnica utiliza a navegação simultânea de muitos robôs, cada robô sabe a posição relativa dos outros robôs, podendo calcular sua posição relativa.

		\item \textbf{Utilização de Pontos de Referência}: Conhecendo pontos de referência que estão distribuídos pelo mapa de navegação, o robô consegue calcular sua posição através da técnica de triangulação.

		\item \textbf{Localização Absoluta com GPS}: A partir desta técnica é fácil obter a posição absoluta do robô em relação à terra. O grande problema desta técnica é a margem de erro presente no sistema de GPS, inviável para navegações internas.

		\item \textbf{Utilização de Bússolas}: É uma técnica interessante para conhecimento da orientação do robô, o que facilita muito na navegação do mesmo. Entretando, as bússolas são muito frágeis a interferências externas, como por exemplo, a proximidade de materiais ferro-magnéticos ou as fugas magnéticas dos motores presentes no próprio robô.

		\item \textbf{Odometria}: Consiste na medição da distância relativa percorrida pelo robô, utilizando sensores presentes nas rodas do mesmo. Necessita do conhecimento do ponto de origem.
		 
	\end{itemize}

	As formas apresentadas anteriormente, para se trabalhar com auto-localização na robótica, possuem características únicas que as adequam para diferentes contextos de navegação. Por exemplo, a Utilização de Mapas é uma técnica bastante útil quando se está trabalhando com um ambiente conhecido e estático, porém, em ambientes mutáveis e não conhecidos, essa estratégia se torna um problema. A Localização Relativa em Grupos funciona muito bem quando a navegação envolve muitos robôs, a qual não necessita de conhecimento prévio do mapa. A Utilização de Pontos de Referência é uma técnica comumente utilizada, a qual é útil quando não se conhece o ambiente de navegação. Entretanto, os pontos de referência, nesse caso, precisam ser conhecidos. 

	Quando se tem ambientes abertos e amplos, a técnica de Localização Absoluta com GPS é a mais utilizada, porem sua margem de erro torna a navegação em ambientes pequenos ou fechados inviável. A Navegação com utilização de Bússulas garante um apoio muito útil para orientação do robô. Entretando, essa técnica gera problemas relacionados a interferências externas, como materiais eletromagnéticos próximos à bússula.
	
	A técnica de Odometria é muito utilizada em navegações curtas, em ambientes com o piso regular e plano. O grande problema desta técnica é adição de erros a cada centímetro percorrido, por meio de derrapagens e falhas no giro das rodas. 

	Desse modo, é fácil perceber que as técnicas possuem características que as adequam a diferentes contextos. Porém, a solução de um problema não precisa utilizar apenas uma delas, podendo integrar características de diferentes técnicas, desde que agregue valor a solução desenvolvida. (REFERENCIAR ESSA AFIRMAÇÃO)

\section{Problema de Pesquisa}

	Sabe-se que todas as técnicas apresentadas acima possuem margens de erros, as quais podem trazer muitos problemas, dependendo do contexto de navegação trabalhado (REFERENCIAR). Tomando como base essa colocação, o presente trabalho buscará unir características de algumas dessas técnicas para desenvolver um método \textbf{eficiente (CUIDADO)} de auto-localização em ambientes fechados. A intenção é tomar como base a técnica de mapeamento do ambiente e auto-localização simultâneos (conhecida como problema de SLAM \textbf{COLOCAR EM EXTENSO E REFERENCIAR}).

	A questão de pesquisa que será discutida durante este trabalho é \textit{"Quais técnicas de auto-localização podem se complementar para apoiar a resolução do problema de SLAM, minimizando as margens de erro durante a navegação, e como deve ser feita a integração dessas técnicas utilizando o Kit de Robótica Educacional da LEGO?"}. \textbf{DUAS QUESTÕES DE PESQUISA, QUEBRAR EM DUAS OU FOCAR EM APENAS UMA???}

\section{Justificativa}

	A utilização da Robótica como uma forma de ensinar programação em escolas e faculdades, a chamada Robótica Educacional, traz benefícios para o aluno. Conforme colocado pelos \textbf{autores [ref1] [ref2]}, alguns desses benefícios são: maior interesse pelos conteúdos estudados em aula, capacidade de trabalhar em grupo \textbf{E...}. A Universidade de Brasília utiliza esta abordagem de ensino/aprendizagem durante a disciplina de Introdução à Robótica Educacional, ministrada pelo professor Dr. Maurício Serrano. Na disciplina, são utilizados Kits de robótica da LEGO \textbf{(OLHAR COMO REFERENCIAE ESSES KITS LA NO DA CAROL)} para desenvolvimento de soluções dos problemas presentes em um tapete de missões. A organização da disciplina se inspira nos campeonatos de robótica, como \cite{ciber-rato} ou \cite{roboBulldozerIV}, por exemplo. Neste tipo de campeonado, a navegação é o quesito mais importante \textbf{(REFERENCIAR ISSO)}, a qual deve possuir a menor margem de erro possível para solucionar as missões. Desse modo, vê-se necessária a seleção e a integração de algumas técnicas de auto-localização para maximizar a precisão desta navegação.
	

	%Levando em consideração a primeira técnica apresentada, a utilização de mapas, \cite{localizacaoEMapeamentoPaulo} mostra a possibilidade do próprio robô construir o mapa do ambiente e utilizá-lo para navegação, simultaneamente. Este problema é conhecido como problema SLAM (da sigla em inglês), que consiste em um robô autônomo iniciar a navegação em uma localização desconhecida e  ambiente desconhecido, construindo um mapa deste ambiente de forma incremental e utilizando-o para calcular sua localização atual.

	%A utilização do SLAM é bastante útil quando não exite o conhecimento prévio do ambiente onde se deseja navegar. Segundo \cite{localizacaoEMapeamentoPaulo}, a resolução do problema SLAM pode ocorrer a partir de diferentes técnicas.


	\section{Objetivos}

	\subsection{Objetivos Gerais} % (fold)
	\label{sub:objetivos_gerais}
	
		Desenvolver um algoritmo para resolução do problema SLAM utilizando o Kit de Robótica Educacional da \textit{LEGO}. \textit{OBS: O objetivo será mais focado no desenvolvimento ou em investigação e pesquisa??}

	% subsection objetivos_gerais (end)

	\subsection{Objetivos Específicos} % (fold)
	\label{sub:objetivos_específicos}

	\begin{itemize}
		\item Resolução do problema de SLAM;
		\item Integração de diferentes técnicas de auto-localização;
		\item Garantia da qualidade da solução (Visão da Engenharia de Software). 
	\end{itemize}
	
	% subsection objetivos_específicos (end)