%\part{Aspectos Gerais}

\chapter[Proposta Inicial]{Proposta Inicial}

\section{Contextualização}
	
	O nascimento da robótica se deu no contexto industrial, onde ferramentas autônomas foram desenvolvidas para executar atividades de forma repetitiva e incansável, maximizando a qualidade dos produtos produzidos e minimizando o custo e tempo para produção dos mesmos \cite{roboticaIndustrial}. Para \cite{roboticaIndustrial}, a palavra robô é derivada da palavra \textit{robota}, de origem eslava, que significa \textit{trabalho forçado}, ou seja, robôs podem ser considerados ferramentas incansáveis que apoiam o trabalho humano. 

	Segundo \cite{localizacaoEMapeamentoPaulo}, a autonomia de um robô é fortemente condicionada pela sua capacidade de perceber o ambiente de navegação, interagindo com o meio e realizando tarefas com o mínimo de precisão. Este mínimo, de acordo com \cite{localizacaoEMapeamentoPaulo}, seria a navegação sem colisão em obstáculos. 

	Para que robôs sejam capazes de navegar em um ambiente desconhecido sem que haja colisão em objetos e obstáculos, os mesmos necessitam de informações sobre este ambiente. Estas informações são adquiridas utilizando sensores. Como foi apresentado por \cite{interacaoRoboAmbiente}, no livro de Robótica Industrial, os sensores possuem o dever de fornecer informações ao sistema de controle do robô sobre distâncias de objetos, posição do robô, contato do robô com objetos, força exercida sobre objetos, cor e textura dos objetos, entre outras.

	Além de obter informações sobre o ambiente, o robô precisa se auto-localizar para processar as informações obtidas e traçar rotas sem colisões até o ponto de destino. Para isso, foram desenvolvidas muitas formas de auto-localização, algumas delas são citadas por \cite{roboBulldozerIV}, como:

	\begin{itemize}
		\item \textbf{Utilização de Mapas}: O robô conhece o mapa onde realizará a navegação à priori, conhecendo os obstáculos e os caminhos possíveis. Possuindo essas informações, o robô irá traçar as rotas mais eficientes para chegar em seu objetivo.

		\item \textbf{Localização Relativa em Grupos}: Esta técnica utiliza a navegação simultânea de muitos robôs, cada robô sabe a posição relativa dos outros robôs, podendo calcular sua posição relativa.

		\item \textbf{Utilização de Pontos de Referência}: Conhecendo pontos de referência que estão distribuídos pelo mapa de navegação, o robô consegue calcular sua posição através da técnica de triangulação.

		\item \textbf{Localização Absoluta com GPS}: A partir desta técnica é fácil obter a posição absoluta do robô em relação à terra. O grande problema desta técnica é a margem de erro presente no sistema de GPS, inviável para navegações internas.

		\item \textbf{Utilização de Bússolas}: É uma técnica interessante para conhecimento da orientação do robô, o que facilita muito na navegação do mesmo. Entretando, as bússolas são muito frágeis a interferências externas, como por exemplo, a proximidade de materiais ferro-magnéticos ou as fugas magnéticas dos motores presentes no próprio robô.

		\item \textbf{Odometria}: Consiste na medição da distância relativa percorrida pelo robô, utilizando sensores presentes nas rodas do mesmo. Necessita do conhecimento do ponto de origem.
		 
	\end{itemize}

	As formas apresentadas anteriormente, para se trabalhar com auto-localização, possuem características únicas que as adequam para diferentes contextos de navegação. Por exemplo, a Utilização de Mapas é uma técnica bastante útil quando se está trabalhando com um ambiente conhecido e estático, porém, em ambientes mutáveis e não conhecidos, essa estratégia se torna um problema. A Localização Relativa em Grupos é a técnica adequada quando a navegação envolve muitos robôs, a qual não necessita de conhecimento prévio do mapa. A Utilização de Pontos de Referência é uma técnica comumente utilizada, a qual é útil quando não se conhece o ambiente de navegação. Entretanto, os pontos de referência, nesse caso, precisam ser conhecidos. 

	Quando se tem ambientes abertos e amplos, a técnica de Localização Absoluta com GPS é a mais utilizada, porem sua margem de erro torna a navegação em ambientes pequenos ou fechados inviável. A Navegação com utilização de Bússulas garante um apoio muito útil para orientação do robô. Entretando, essa técnica gera problemas relacionados a interferências externas, como materiais eletromagnéticos próximos à bússula.
	
	A técnica de Odometria é muito utilizada em navegações curtas, em ambientes com o piso regular e plano. O grande problema desta técnica é adição de erros a cada centímetro percorrido, por meio de derrapagens e falhas no giro das rodas. 

	Desse modo, é fácil perceber que cada técnica possui características que se adaptam melhor para diferentes situações. Durante este trabalho, a situação atacada é a navegação autônoma de robôs simples e baratos, como os kits de robótica da LEGO. Os quais possuem poucas opções de sensores e características bem limitadas \cite{drawLegoRobot}.

\section{Problema de Pesquisa}

	Todas as técnicas de auto-localização apresentadas na seção anterior já foram testadas, comparadas e refatoradas em diferentes contextos. Desde a navegação marítima até questões relacionadas a tecnologia aeroespacial \cite{localizacaoEMapeamentoPaulo}. Porém, sabe-se que, nestes contextos, o \textit{hardware} utilizado para navegação era de alta tecnologia, possuindo processadores de altíssimo desempenho e sensores bem precisos.

	Já em um contexto educacional, a infraestrutura disponível nem sempre engloba os critérios necessários para aplicação das técnicas de auto-localização. Um exemplo disso é a utilização, por \cite{localizacaoEMapeamentoPaulo}, de uma câmera omnidirecional para obter informações sobre o ambiente. Alem da dificuldade de aquisição do \textit{hardware}, a complexidade em trabalhar com este tipo de análise inviabiliza a utilização da mesma em um sistema educacional onde o objetivo é ensinar programação a partir da robótica.

	Desse modo, vê-se a necessidade da adaptação de técnicas de auto-localização para o contexto da Robótica Educacional, utilizando os kits de robótica \textit{Mindstorms} da LEGO. Onde a capacidade de processamento, a precisão dos sensores e dos atuadores é limitada. A intenção é tomar como base a técnica de mapeamento do ambiente e auto-localização simultâneos (conhecida como problema de SLAM - \textit{Simultaneous Location and Mapping}) \cite{slamProblem}.

	A questão de pesquisa que será discutida durante este trabalho é \textit{"Como solucionar o problema de SLAM no contexto limitado da Robótica Educacional?"}.

\section{Justificativa}

	A utilização da Robótica como uma forma de ensinar programação em escolas e faculdades, a chamada Robótica Educacional, traz alguns benefícios para o aluno. Conforme colocado pelos autores \cite{teachingWithRoboticKit}, \cite{roboticEducationBasedLego}, \cite{roboticaEducacionalAulasMatematica} e \cite{evaluationRoboticEducationScale}, alguns desses benefícios são: maior interesse pelos conteúdos estudados em aula, capacidade de trabalhar em grupo, aplicação prática do conhecimento teórico e multidisciplinaridade. A Universidade de Brasília utiliza esta abordagem de ensino/aprendizagem durante a disciplina de Introdução à Robótica Educacional, ministrada pelo professor Dr. Maurício Serrano. Na disciplina, são utilizados Kits de robótica Mindstorms, da Lego, para desenvolvimento de soluções dos problemas presentes em um tapete de missões. A organização da disciplina se inspira nos campeonatos de robótica, como \cite{ciber-rato} ou \cite{roboBulldozerIV}, por exemplo. Neste tipo de campeonado, a navegação é o quesito mais importante \cite{ciber-rato}, a qual deve possuir a menor margem de erro possível para solucionar as missões.

	As missões utilizadas durante a disciplina são referentes a problemas recorrentes no contexto da robótica mundial. Porém, a solução dos mesmos é uma adaptação das técnicas existentes para o contexto limitado da disciplina, onde são utilizadas ferramentas presentes no kit \textit{Mindstorms} da Lego. Esta adaptação exige um conhecimento amplo sobre a técnica, para que o estudante possa identificar características relevantes e adaptá-las de acordo com o \textit{hardware} disponível.

	%Boa parte dos problemas encontrados pelos estudantes durante a realização de uma missão da disciplina se refere a auto-localização do robô. Desse modo, a identificação de uma técnica que solucione o problema de auto-localização utilizando as ferramentas disponíveis na disciplina trará novas opções aos alunos, que serão inseridos no contexto do problema de SLAM.

	\section{Objetivos}

	\subsection{Objetivos Gerais} % (fold)
	\label{sub:objetivos_gerais}
	
		Adaptar técnicas de resolução do problema de SLAM para o contexto da Robótica Educacional, utilizando os kits de robótica \textit{Mindstorms} da Lego.

	% subsection objetivos_gerais (end)

	\subsection{Objetivos Específicos} % (fold)
	\label{sub:objetivos_específicos}

		O Objetivo Geral pode ser dividido em quatro objetivos específicos, que vão desde a solução do problema até a adaptação da mesma pro contexto educacional, como são apresentados a seguir.
		 
	\begin{itemize}
		\item Resolução do problema de SLAM;
		\item Integração de diferentes técnicas de auto-localização;
		\item Garantia da qualidade da solução (Visão da Engenharia de Software);
		\item Simplificação da solução (Contexto educacional). 
	\end{itemize}
	
	% subsection objetivos_específicos (end)