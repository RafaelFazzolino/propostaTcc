\chapter[Metodologia]{Metodologia}

Este trabalho segue uma metodologia de pesquisa teórica, exploratória e com uma abordagem qualitativa, onde são selecionadas, comparadas e analisadas técnicas de resolução do problema de SLAM. De acordo com \cite{metodologiaCientifica}, a pesquisa exploratória tem como objetivo ampliar os conhecimentos do pesquisador sobre o tema. O uso da pesquisa exploratória, neste trabalho, se dá pela necessidade do amplo conhecimento sobre as técnicas de resolução do problema de SLAM, com o intuito de realizar adaptações das mesmas para o contexto educacional.

Com o objetivo de identificar o máximo de técnicas possível, com os mais diferentes tipos de \textit{hardwares}, será realizada uma longa pesquisa bibliográfica sobre os temas \textit{Auto-localização na Robótica, o Problema de SLAM e as características da Robótica Educacional}. As principais fontes de dados utilizadas são as bases da CAPES, IEEE e Scopus.

A primeira fase deste trabalho é focada em estabelecer pilares teóricos que sustentem um projeto de adaptação de técnicas durante a segunda fase. Esta primeira fase pode ser sub-dividida em três atividades: \textit{Realizar Pesquisa Bibliográfica}, \textit{selecionar Técnicas} e \textit{Propor Adaptações}. As mesmas estão distribuídas de acordo com o cronograma disposto na tabela \ref{tab:cronograma}.

\begin{table}[http]
	\centering
	\caption{Cronograma TCC 1}
	\label{tab:cronograma}
	\begin{tabular}{ccccc}
		\hline
		\multicolumn{1}{|c|}{\textbf{Cronograma}}             & \multicolumn{1}{c|}{\textbf{Março}} & \multicolumn{1}{c|}{\textbf{Abril}} & \multicolumn{1}{c|}{\textbf{Maio}} & \multicolumn{1}{c|}{\textbf{Junho}} \\ \hline
		\multicolumn{1}{|c|}{Realizar Pesquisa Bibliográfica} & \multicolumn{1}{c|}{X}              & \multicolumn{1}{c|}{X}              & \multicolumn{1}{c|}{X}             & \multicolumn{1}{c|}{X}              \\ \hline
		\multicolumn{1}{|c|}{Selecionar técnicas}             & \multicolumn{1}{c|}{}               & \multicolumn{1}{c|}{X}              & \multicolumn{1}{c|}{X}             & \multicolumn{1}{c|}{}               \\ \hline
		\multicolumn{1}{|c|}{Propor adaptação}                & \multicolumn{1}{c|}{}               & \multicolumn{1}{c|}{}               & \multicolumn{1}{c|}{X}             & \multicolumn{1}{c|}{X}              \\ \hline
		\multicolumn{1}{l}{}                                  & \multicolumn{1}{l}{}                & \multicolumn{1}{l}{}                & \multicolumn{1}{l}{}               & \multicolumn{1}{l}{}                \\
		\multicolumn{1}{l}{}                                  & \multicolumn{1}{l}{}                & \multicolumn{1}{l}{}                & \multicolumn{1}{l}{}               & \multicolumn{1}{l}{}                \\
		\multicolumn{1}{l}{}                                  & \multicolumn{1}{l}{}                & \multicolumn{1}{l}{}                & \multicolumn{1}{l}{}               & \multicolumn{1}{l}{}               
	\end{tabular}
\end{table}

Quanto aos procedimentos de desenvolvimento, durante a fase de adaptação de técnicas de resolução do problema de SLAM, a metodologia seguida será baseada no \textit{Scrum}, utilizando sprints de 2 semanas. O desenvolvimento se dará com base em provas de conceito que buscarão sustentar a viabilidade das adaptações propostas.
