\chapter[Metodologia]{Metodologia}

Este trabalho segue uma metodologia de pesquisa teórica, exploratória e com uma abordagem qualitativa, onde são selecionadas, comparadas e analisadas técnicas de resolução do problema de SLAM. Segundo \cite{metodologiaCientifica}, a pesquisa exploratória tem como objetivo ampliar os conhecimentos do pesquisador sobre o tema. O uso da pesquisa exploratória, neste trabalho, se dá pela necessidade do amplo conhecimento sobre as técnicas de resolução do problema de SLAM, com o intuito de realizar adaptações das mesmas para o contexto educacional.

Com o objetivo de identificar o máximo de técnicas possível, com os mais diferentes tipos de \textit{hardware}, será realizada uma Revisão Sistemática sobre o tema \textit{Auto-localização na Robótica e o Problema de SLAM}, alem de pesquisa bibliográfica sobre Robótica Educacional. As principais fontes de dados utilizadas são as bases da CAPES, IEEE e Scopus.